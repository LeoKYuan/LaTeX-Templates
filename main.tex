% LaTeX Homework Assignment / Report / Essay Template (Version 1.0)
% Author: Y. YUAN
% Email: yuanym2018@foxmail.com
% QQ: 475378169

\documentclass[a4paper,10pt]{article} % Uses article class in A4 format

% Formatting

\setlength{\parskip}{0pt}
\setlength{\parindent}{0pt}
\setlength{\voffset}{-15pt}
\setlength{\headheight}{0pt}

% Package and other settings


\usepackage[a4paper, margin=2.5cm]{geometry} % Sets margin to 2.5cm for A4 Paper
\usepackage[onehalfspacing]{setspace} % Sets Spacing to 1.5

\usepackage[T1]{fontenc} % Use European encoding

%\usepackage[utf8]{inputenc} % Use UTF-8 encoding. No need if using XeLaTeX.
\usepackage{charter} % Use the Charter font
\usepackage{microtype} % Slightly tweak font spacing for aesthetics

\usepackage[english]{babel} % Language hyphenation and typographical rules

\usepackage{amsthm, amsmath, amssymb} % Mathematical typesetting
\usepackage{marvosym, wasysym} % More symbols
\usepackage{float} % Improved interface for floating objects
\usepackage[final, colorlinks = true, 
            linkcolor = black, 
            citecolor = black,
            urlcolor = black]{hyperref} % For hyperlinks in the PDF
\usepackage{graphicx, multicol} % Enhanced support for graphics
\usepackage{xcolor} % Driver-independent color extensions
\usepackage{rotating} % Rotation tools
\usepackage{listings, style/lstlisting} % Environment for non-formatted code, !uses style file!
\usepackage{pseudocode} % Environment for specifying algorithms in a natural way
\usepackage{style/avm} % Environment for f-structures, !uses style file!
\usepackage{booktabs} % Enhances quality of tables

\usepackage{tikz-qtree} % Easy tree drawing tool
\tikzset{every tree node/.style={align=center,anchor=north},
         level distance=2cm} % Configuration for q-trees
\usepackage{style/btree} % Configuration for b-trees and b+-trees, !uses style file!

\usepackage{titlesec} % Allows customization of titles

% Title Format
\renewcommand\thesection{\arabic{section}.} % Arabic numerals for the sections
\titleformat{\section}[block]{\Large\bfseries}{\thesection}{1em}{}
\renewcommand\thesubsection{\alph{subsection})} % Alphabetic numerals for subsections
\titleformat{\subsection}{\large\itshape\bfseries}{\thesubsection}{1em}{}
\renewcommand\thesubsubsection{\roman{subsubsection}.} % Roman numbering for subsubsections
\titleformat{\subsubsection}{\large}{\thesubsubsection}{1em}{}

\usepackage[all]{nowidow} % Removes widows


\usepackage{csquotes} % Context sensitive quotation facilities

\usepackage[ddmmyyyy]{datetime} % Uses YEAR-MONTH-DAY format for dates
%\renewcommand{\dateseparator}{-} % Sets dateseparator to '-'

\usepackage{fancyhdr} % Headers and footers

%Other packages
\usepackage{framed} %Generating a Textbox
\usepackage{cite} %Package for citation

\pagestyle{fancy} % All pages have headers and footers
\fancyhead{}\renewcommand{\headrulewidth}{0pt} % Blank out the default header
\fancyfoot[L]{\textbf{Email: yuanym2018@foxmail.com}} % Custom footer text
\fancyfoot[R]{\includegraphics[width=0.25\columnwidth]{logo}} % Custom footer text
\fancyfoot[C]{\thepage} % Custom footer text

\newcommand{\note}[1]{\marginpar{\scriptsize \textcolor{red}{#1}}} % Enables comments in red on margin


%%%%%%%%%%%%%%%%%%%%%%%%%%%%%%%%%%%%%%%%%%%%%%%%%%%%%%%%%%%
%%%%%%%%%%%%%%%%%%%%%%%%%%%%%%%%%%%%%%%%%%%%%%%%%%%%%%%%%%%
%%%%%%%%%%%%%%%%%%%%%%%%%%%%%%%%%%%%%%%%%%%%%%%%%%%%%%%%%%%


\begin{document}

% TOP LOGO

\begin{center}
    \includegraphics[width=0.6\columnwidth]{Logo10.png}
\end{center}

%BIG TITLE

\begin{minipage}{0.995\textwidth}
    \centering
    \vspace{0.5em}
    \rule[0em]{1\columnwidth}{0.3mm}
    \vspace{0em}
    \huge % Title text size
    Assignment \#3 %Title
    \rule[0.5em]{1\columnwidth}{0.3mm}
\end{minipage}

% Information

\begin{minipage}{0.495\textwidth} % Left side of information
\raggedright
\textbf{Course:} Your Cource\\
\textbf{Cource ID:} The Code of Cource\\
\textbf{Professor:} Your Instructor
\end{minipage}
\begin{minipage}{0.495\textwidth} % Right side of information
\raggedleft
\textbf{Name:} Y. YUAN\\
\textbf{SID:} Your Student ID\\
\textbf{Date:} \today \\
\end{minipage}
\rule[2em]{1\columnwidth}{1mm}

% Main Body

\section*{Question}

\begin{framed}
    Consider the pseudo-NMOS inverter shown in Fig. 1, with transistor parameters given in Table 1.

    (a) Find $V_{OL}$ and $V_{OH}$.

    (b) Find the switching threshold $V_M$.

    (c) If the same transistors are used to build a static CMOS inverter, i.e. the gate of PMOS is connected to $V_{in}$ instead of GND, what is the new switching threshold $V_M$?

    \begin{center}
        \includegraphics[width=0.7\columnwidth]{1.png}
    \end{center}
\end{framed}

\subsection*{Answer.}

\subsubsection*{(a)}

Let $V_{in}=0\,V$.

Then $M_1$ is off, and $M_2$ is on. 

$$\Rightarrow \qquad V_{OH} = V_{Out} = V_{DD} = 2.5\, V$$

Let $V_{in}=V_{OH} = 2.5 \, V$.

Then both $M_1$ and $M_2$ are on. The drain current of each transistor can be expressed as:

\begin{equation*}
    \begin{cases}
        I_{D1} = k^{\prime}_n (W/L)_n (V_{GTn}V_{Out} - \frac{1}{2}V_{Out}^2)(1+\lambda_n V_{Out})\\
        I_{D2} = k^{\prime}_p (W/L)_p (V_{GTp}V_{DSATp} - \frac{1}{2}V_{DSATp}^2)(1+\lambda_p (V_{Out} - V_{DD})))
    \end{cases}
\end{equation*}

According to Kirchoff Current Law, $I_{D1} + I_{D2}=0$.

Given: $k^{\prime}_n = 115 \times 10^{-6} \, A/V^2$, $k^{\prime}_p = -30\times 10^{-6} \, A/V^2$, $(W/L)_n = 8$, $(W/L)_p = 2$, $V_{GTn} = V_{in} - V_{T0n} = 2.07 \, V$, $V_{GTp} = 0 \, V - 2.5 \, V - V_{T0p} = -2.1 \, V$, $V_{DSATp} = -1 \, V$, $\lambda_n = 0.06 \, V^{-1}$, $\lambda_p = -0.1 \, V^{-1}$.

Solve these equations, we have: $V_{Out} = 0.0634 \, V$. Then: $$V_{OL} = V_{Out} = 0.0634 \, V$$

\subsubsection*{(b)}

The drain current of each transistor can be expressed as:

\begin{equation*}
    \begin{cases}
        I_{D1} = k^{\prime}_n (W/L)_n (V_{GTn}V_{DSATn} - \frac{1}{2}V_{DSATn}^2)(1+\lambda_n V_{M})\\
        I_{D2} = k^{\prime}_p (W/L)_p (V_{GTp}V_{DSATp} - \frac{1}{2}V_{DSATp}^2)(1+\lambda_p (V_{M} - V_{DD})))
    \end{cases}
\end{equation*}

According to Kirchoff Current Law, $I_{D1} + I_{D2}=0$.

Given: $k^{\prime}_n = 115 \times 10^{-6} \, A/V^2$, $k^{\prime}_p = -30\times 10^{-6} \, A/V^2$, $(W/L)_n = 8$, $(W/L)_p = 2$, $V_{GTn} = V_{M} - V_{T0n}$, $V_{GTp} = 0 \, V - V_{DD} - V_{T0p} = -2.1 \, V$, $V_{DSATn} = 0.6 \, V$, $V_{DSATp} = -1 \, V$, $\lambda_n = 0.06 \, V^{-1}$, $\lambda_p = -0.1 \, V^{-1}$.

Solve these equations, we have: $$V_{M} = 0.921 \, V$$.

\subsubsection*{(c)}

The drain current of each transistor can be expressed as:

\begin{equation*}
    \begin{cases}
        I_{D1} = k^{\prime}_n (W/L)_n (V_{GTn}V_{DSATn} - \frac{1}{2}V_{DSATn}^2)(1+\lambda_n V_{M})\\
        I_{D2} = k^{\prime}_p (W/L)_p (V_{GTp}V_{DSATp} - \frac{1}{2}V_{DSATp}^2)(1+\lambda_p (V_{M} - V_{DD})))
    \end{cases}
\end{equation*}

According to Kirchoff Current Law, $I_{D1} + I_{D2}=0$.

Given: $k^{\prime}_n = 115 \times 10^{-6} \, A/V^2$, $k^{\prime}_p = -30\times 10^{-6} \, A/V^2$, $(W/L)_n = 8$, $(W/L)_p = 2$, $V_{GTn} = V_{M} - V_{T0n}$, $V_{GTp} = V_M -  V_{DD} - V_{T0p}$, $V_{T0n} = 0.43\, V$, $V_{T0p} = -0.4\, V$, $V_{DSATn} = 0.6 \, V$, $V_{DSATp} = -1 \, V$, $\lambda_n = 0.06 \, V^{-1}$, $\lambda_p = -0.1 \, V^{-1}$.

Solve these equations, we have: $$V_{M} = 0.824 \, V$$.
\bigskip

% REFERENCE

asdfghjkl \cite{latexIn24Hours} %This line is just an example of citation. Remove it when using this template.

\bibliographystyle{apalike}
\bibliography{main.bib}
\end{document}
